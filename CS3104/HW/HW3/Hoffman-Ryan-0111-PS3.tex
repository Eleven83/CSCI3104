\documentclass[12pt]{article}
\setlength{\oddsidemargin}{0in}
\setlength{\evensidemargin}{0in}
\setlength{\textwidth}{6.5in}
\setlength{\parindent}{0in}
\setlength{\parskip}{\baselineskip}

\usepackage{amsmath,amsfonts,amssymb}
\usepackage{graphicx}
\usepackage{fancyhdr}
\usepackage{ulem}
\pagestyle{fancy}
\usepackage[all]{xy}
\setlength{\headheight}{15pt}

\begin{document}

CSCI 3104-300 Summer 2019 \hfill Problem Set 3 \\
Hoffman, Ryan \\
01/11

\hrulefill

\begin{enumerate}

	\item (10 points) For parts~\eqref{qs:1} and~\eqref{qs:2}, justify your answers in terms of deterministic QuickSort, and 
	for part~\eqref{qs:3}, refer to Randomized QuickSort. In both cases, refer to the versions of the algorithms given in 
	lecture (you can refer to the moodle lecture notes).
	\begin{enumerate}
	\item \label{qs:1} What is the asymptotic running time of QuickSort when every element of the input $A$ is identical, i.e., 
	for $1\leq i,j \leq n$, $A[i] = A[j]$?\par
	\textbf{Solution:}\par
	If all of the elements are the same, then array A is just an array with one element and is therefore already sorted
	trivially. This falls under a worst case scenario where, from our notes, $T(n)=\Theta(n^2)$.\par
	\item \label{qs:2} Let the input array $A = [9, 7, 5, 11, 12, 2, 14, 3, 10, 6]$. What is the number of times a 
	comparison is made to the element with value 3?\par
		\textbf{Solution:}\par
		According to my interpretation of the pseudocode and a deterministic QuickSort python program that I ran, 
		the number of times that a comparison is made to the value 3 is three(3).\par
	\item \label{qs:3} How many calls are made to {\tt random-int} in (i) the worst case and (ii) the best case? Give your 
	answers in asymptotic notation.\par
		\textbf{Solution:}\par
		If we let $N(n)$ be the number of times the {\tt random-int} function is called on the array, A, of size $n$, then we only
		have two cases to consider:\par
		\begin{enumerate}
			\item When the partition creates two equal size subarrays/subproblems we define $N(n)=2N(\frac{n}{2})+1$. 
			This yields $N(n)=\Theta(n)$.
			\item When the partition creates a subproblem of size $n-1$, we define $N(n)=N(n-1)+1$. 
			However, this also yields $N(n)=\Theta(n)$. 
		\end{enumerate}\par
		Therefore, I conclude based on class notes and the above definition of $N(n)$, that both cases require
		$\Theta(n)$ recursive calls to {\tt random-int} and there is no best or worst case.\par
	\end{enumerate}
	
	\newpage

	\item (20 points) Solve the following recurrence relations using any of the following methods:\ unrolling, substitution, or 
	recurrence tree (include tree diagram). For each case, show your work.
    \begin{enumerate}
    	\item $T(n) = T(n-2) + C$ if $n>1$, and $T(n) = C$ otherwise\par
		\textbf{Solution:}\par
			$T(n) = T(n-2) + C$\par
			$= T(n-4) + C + C$\par
			$= T(n-6) + C + C + C$\par
			$T(n) = T(n-2k) + Ck$\par
			Now, $T(0)$,$T(1) = C$\par
			If even: $n-2k = 0 \Rightarrow k = \frac{n}{2}$\par
			Then $T(n) = T(n-2\frac{n}{2})+\frac{n}{2}C$\par
			$=T(0) + \frac{n}{2}C$\par
			If odd: $n-2k = 1 \Rightarrow k = \frac{n-1}{2}$\par
			$T(n) = T(n-2\frac{n-1}{2})+\frac{n-1}{2}C$\par
			$= T(1)+\frac{n-1}{2}C$\par
    	\item $T(n) = 3T(n-1) + 1$ if $n>1$, and $T(1) = 3$\par
		\textbf{Solution:}\par
		$T(n) = 3T(n-)+1$\par
			$= 3[3T(n-2) + 1] + 1$\par
			$= 9T(n-2) + 3 + 1$\par
			$= 9[3T(n-3)+1]+3+1$\par
			$= 3^3[3T(n-4)+1]+3^2 +3^1 +1$\par
			$= 3^kT(n-k)+3^{k-1}+3^{k-2}+3^{k-3}\dots +1$\par
			$T(n) = \frac{3^{n}+3^{n-1}-1}{2}$\par
	\newpage
    	\item $T(n) = T(n-1)+2^{n}$ if $n>1$, and $T(1) = 2$\par
		\textbf{Solution:}\par
		$T(n) = T(n-1)+2^{n}$\par
		$= [T(n-2)+2^{n-1}]+2^{n}$\par
		$= t(n-3)+2^{n-2}+2^{n-1}+2^{n}$\par
		$= T(n-4)+2^{n-3}+2^{n-2}+2^{n-1}+2^{n}$\par
		$= T(n-k)+2^{n-k-1}+\dots+2^{n}$\par
		I was not able to go any further at this point. \par
    	\item $T(n) = T(\sqrt{n}) + 1$ if $n\geq2$ , and $T(n) = 0$ otherwise\par
		\textbf{Solution:}\par
		$T(n) = T(\sqrt{n}) + 1$\par
		$= T(n^{\frac{1}{2}})+1$\par
		$= T(n^{\frac{1}{4}})+1+1$\par
		$= T(n^{\frac{1}{8}})+1+1+1$\par
		I was not able to go any further at this point.\par
    \end{enumerate}

    \newpage

    \item
	(20 points) Use the Master Theorem to solve the following recurrence relations. 
	For each recurrence, either give the asympotic solution using the Master Theorem (state which case), or 
	else state the Master Theorem doesn't apply.
    \begin{enumerate}
		\item $T(n) = T(\frac{3n}{4}) + 2$\par
		\textbf{Solution:}\par
		a=1, b=$\frac{4}{3}$, and d=2 $\Rightarrow T(n)=\Theta(1)$\par 
		
		\item $T(n) = 3T(\frac{n}{4}) + nlgn$\par
		\textbf{Solution:}\par
		a=3, b=4, and d=$nlogn \Rightarrow T(n)=\Theta(nlogn)$\par

		\item $T(n) = 8T(\frac{n}{3}) + 2^n$\par
		\textbf{Solution:}\par
		a=8, b=3, and d=2 $\Rightarrow T(n)=\Theta(2^{n})$, by case (3).\par 
		
		\item $T(n) = T(\frac{n}{2}) + T(\frac{n}{4}) + n^2$\par
		\textbf{Solution:}\par
		a=2, b=2, and d=$n^{2}$ $\Rightarrow $Master Theorem does not apply.\par 
		
		\item $T(n) = 100T(\frac{n}{42}) + \lg n$\par
		\textbf{Solution:}\par
		a=100, b=42, and d=$n^{2}$ $\Rightarrow T(n)=\Theta(lgn)$\par
	\end{enumerate}

    \newpage

	\item (30 points)
	Professor Trelawney has acquired \textit{n} enchanted crystal balls, of dubious origin and dubious reliability. 
	Trelawney needs your help to identify which crystal balls are accurate and which are inaccurate. She has constructed 
	a strange contraption that fits over two crystal balls at a time to perform a test. When the contraption is activated, 
	each crystal ball glows one of two colors depending on whether the other crystal ball is accurate or not. 
	An accurate crystal ball always glows correctly according to whether the other crystal ball is 
	accurate or not, but the glow of an inaccurate crystal ball glows the opposite color of what the other crystal ball is 
	(i.e. If the other crystal ball is accurate, it will glow red. If the other crystal ball is inaccurate it will glow green). 
	You quickly notice that there are two possible test outcomes:
	\begin{small}
    	\begin{center}
        	\begin{tabular}{ccll}
        	crystal ball $i$ glows & crystal ball $j$ glows & &  \\
        	\hline
        	red & red & $\implies$ & at least one is inaccurate \\
        	green & green & $\implies$ & both are accurate, or both inaccurate \\
        	\end{tabular}
    	\end{center}
	\end{small}

	\begin{enumerate}
		\item Prove that if $n/2$ or more crystal balls are inaccurate, Professor Trelawney cannot necessarily determine which 
		crystal balls are tainted using any strategy based on this kind of pairwise test.\par
		\textbf{Solution:}\par
		Using CLRS Chip Testing Method as a guide, we see that, due to symmetric behaviors exhibited on the sets of accurate
		and inaccurate crystal balls, they basically cancel each other out. Therefore it is not possible to determine which
		crystal balls are tainted or which are accurate.
		\item \label{chips:b} \sout{Consider the problem of finding a single good crystal ball from among the $n$ crystal balls, and 
		suppose Professor Trelawney knows that more than $n/2$ of the crystal balls are accurate, but not which ones. 
		Prove that $\lfloor n/2\rfloor$ pairwise tests are sufficient to reduce the problem to one of nearly half the size.}\par
		\textbf{Solution:}\par
		This question has been eliminated.
		\item Now, under the same assumptions as part~\eqref{chips:b}, prove that all of the accurate crystal balls can be 
		identified with $\Theta(n)$ pairwise tests. Give and solve the recurrence that describes the number of tests.\par
		\textbf{Solution:}\par
		Using the last sentence from part (b), we find a single accurate crystal ball using the recurrence relation:\par
		$T(n)\leq T(\lceil n/2\rceil)+\lfloor n/2\rfloor$\par
		and, by the Master theorem, $\Rightarrow T(n)=\mathcal{O}(n)$. Then we add the remaining $n-1$ crystal balls to get $T(n)=\mathcal{O}(n)+n-1 =\Theta(n)$ tests.
\newpage

	\end{enumerate}

    \newpage

	\item (10 points) Harry needs your help breaking into a dwarven lock box. 
	The lock box projects an array $A$ consisting of $n$ integers $A[1], A[2], \dots , A[n]$ and has you enter in a 
	two-dimensional $n\times n$ array $B$ -- to open the box -- in which $B[i,j]$ (for $i<j$) contains the sum of 
	array elements $A[i]$ through $A[j]$, i.e., $B[i,j] = A[i]+A[i+1]+\dots+A[j]$. 
	(The value of array element $B[i,j]$ is left unspecified whenever $i\geq j$, so it doesn't matter what the output is for these values.)\\
	Harry suggests the following simple algorithm to solve this problem:\par
	\begin{small}
    	\begin{verbatim}
        	dwarvenLockBox(A) {
        	   for i = 1 to n {
        	      for j = i+1 to n {
        	         s = sum(A[i..j])       // look very closely here
        	         B[i,j] = s
        	}}}
    	\end{verbatim}
	\end{small}

	\begin{enumerate}
		\item For some function $g$ that you should choose, give a bound of the form $\Omega(g(n))$ on the running time of this 
		algorithm on an input of size $n$ (i.e., a bound on the number of operations performed by the algorithm).\par
		\textbf{Solution:}\par
		The outer $for$ loop runs $n$ times and the inner loop will run a maximum of $n$ times for each iteration of 
		the outer loop, which then sums the elements in $A$ and stores them in $B$ giving us a total of $\Omega(n^{3})$\par 
		\item For this same function $g$, show that the running time of the algorithm on an input of size $n$ is also $O(g(n))$. 
		(This shows an asymptotically tight bound of $\Theta(g(n))$ on the running time.)\par
		\textbf{Solution:}\par
		The same function g must have the same upper bound as the lower bound$\Omega(n^{3})$ which yields an asymptotically tight bound 
		of $\Theta(g(n))$ on the running time.\par
	\end{enumerate}

    \newpage

	\item (20 points) Consider the following strategy for choosing a pivot element for \par 
	the {\tt Partition} subroutine of QuickSort, applied to an array $A$.\par
    \begin{itemize}
    	\item Let $n$ be the number of elements of the array $A$.\par
    	\item If $n\leq 24$, perform an Insertion Sort of $A$ and return.\par
    	\item Otherwise:
    	\begin{itemize}
    		\item Choose $2\lfloor n^{(1/2)} \rfloor$ elements at random from $n$; let $S$ be the new list with the chosen elements.
    		\item Sort the list $S$ using Insertion Sort and use the median $m$ of $S$ as a pivot element.
    		\item Partition using $m$ as a pivot.
    		\item Carry out QuickSort recursively on the two parts.
		\end{itemize}
    \end{itemize}
    \begin{enumerate}
    	\item How much time does it take to sort $S$ and find its median? Give a $\Theta$ bound.\par
		\textbf{Solution:}\par
		I was unable to formulate a solution for the last question.
    	\item If the element $m$ obtained as the median of $S$ is used as the pivot, what can we say about the sizes of the two partitions of the array $A$?\par
		\textbf{Solution:}\par
		I was unable to formulate a solution for the last question.
    	\item Write a recurrence relation for the worst case running time of QuickSort with this pivoting strategy.\par
		\textbf{Solution:}\par
		I was unable to formulate a solution for the last question.
    \end{enumerate}

\end{enumerate}

\end{document}
