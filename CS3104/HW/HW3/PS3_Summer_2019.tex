\documentclass[12pt]{article}
\setlength{\oddsidemargin}{0in}
\setlength{\evensidemargin}{0in}
\setlength{\textwidth}{6.5in}
\setlength{\parindent}{0in}
\setlength{\parskip}{\baselineskip}

\usepackage{amsmath,amsfonts,amssymb}
\usepackage{graphicx}
\usepackage{fancyhdr}
\pagestyle{fancy}
\usepackage[all]{xy}

\begin{document}

\lhead{{\bf CSCI 3104 \\ Problem Set 3 (110 points)} }
\rhead{{\bf Instructor\ Buxton\\ Spring 2019, CU-Boulder}}
\renewcommand{\headrulewidth}{0.4pt}

% 10+20+20+30+10+20 = 110 points possible

\begin{enumerate}

	% EASY PROBLEMS
	\item (10 points) For parts~\eqref{qs:1} and~\eqref{qs:2}, justify your answers in terms of deterministic QuickSort, and for part~\eqref{qs:3}, refer to Randomized QuickSort. In both cases, refer to the versions of the algorithms given in lecture (you can refer to the moodle lecture notes).
	\begin{enumerate}
	\item \label{qs:1} What is the asymptotic running time of QuickSort when every element of the input $A$ is identical, i.e., for $1\leq i,j \leq n$, $A[i] = A[j]$?
	\item \label{qs:2} Let the input array $A = [9, 7, 5, 11, 12, 2, 14, 3, 10, 6]$. What is the number of times a comparison is made to the element with value 3?
	\item \label{qs:3} How many calls are made to {\tt random-int} in (i) the worst case and (ii) the best case? Give your answers in asymptotic notation.
	\end{enumerate}

	\newpage

	% MEDIUM PROBLEM
    \item (20 points) Solve the following recurrence relations using any of the following methods:\ unrolling, substitution, or recurrence tree (include tree diagram). For each case, show your work.
    \begin{enumerate}
    	\item $T(n) = T(n-2) + C$ if $n>1$, and $T(n) = C$ otherwise
    	\item $T(n) = 3T(n-1) + 1$ if $n>1$, and $T(1) = 3$
    	\item $T(n) = T(n-1)+2^{n}$ if $n>1$, and $T(1) = 2$
    	\item $T(n) = T(\sqrt{n}) + 1$ if $n\geq2$ , and $T(n) = 0$ otherwise
    \end{enumerate}

    \newpage

    \item
    (20 points) Use the Master Theorem to solve the following recurrence relations. For each recurrence, either give the asympotic solution using the Master Theorem (state which case), or else state the Master Theorem doesn't apply.
    \begin{enumerate}
    	\item $T(n) = T(\frac{3n}{4}) + 2$
    	\item $T(n) = 3T(\frac{n}{4}) + nlgn$
    	\item $T(n) = 8T(\frac{n}{3}) + 2^n$
    	\item $T(n) = T(\frac{n}{2}) + T(\frac{n}{4}) + n^2$
    	\item $T(n) = 100T(\frac{n}{42}) + \lg n$
	\end{enumerate}

    \newpage

	% HARD PROBLEM
	\item (30 points)
	Professor Trelawney has acquired \textit{n} enchanted crystal balls, of dubious origin and dubious reliability. Trelawney needs your help to identify which crystal balls are accurate and which are inaccurate. She has constructed a strange contraption that fits over two crystal balls at a time to perform a test. When the contraption is activated, each crystal ball glows one of two colors depending on whether the other crystal ball is accurate or not. An accurate crystal ball always glows correctly according to whether the other crystal ball is accurate or not, but the glow of an inaccurate crystal ball glows the opposite color of what the other crystal ball is (i.e. If the other crystal ball is accurate, it will glow red. If the other crystal ball is inaccurate it will glow green). You quickly notice that there are two possible test outcomes:

	\begin{small}
    	\begin{center}
        	\begin{tabular}{ccll}
        	crystal ball $i$ glows & crystal ball $j$ glows & &  \\
        	\hline
        	red & red & $\implies$ & at least one is inaccurate \\
        	green & green & $\implies$ & both are accurate, or both inaccurate \\
        	\end{tabular}
    	\end{center}
	\end{small}

	\begin{enumerate}
    	\item Prove that if $n/2$ or more crystal balls are inaccurate, Professor Trelawney cannot necessarily determine which crystal balls are tainted using any strategy based on this kind of pairwise test.

    	\item \label{chips:b} Consider the problem of finding a single good crystal ball from among the $n$ crystal balls, and suppose Professor Trelawney knows that more than $n/2$ of the crystal balls are accurate, but not which ones. Prove that $\lfloor n/2\rfloor$ pairwise tests are sufficient to reduce the problem to one of nearly half the size.

    	\item Now, under the same assumptions as part~\eqref{chips:b}, prove that all of the accurate crystal balls can be identified with $\Theta(n)$ pairwise tests. Give and solve the recurrence that describes the number of tests.
	\end{enumerate}

    \newpage

	% MEDIUM PROBLEM
	\item (10 points) Harry needs your help breaking into a dwarven lock box. The lock box projects an array $A$ consisting of $n$ integers $A[1], A[2], \dots , A[n]$ and has you enter in a two-dimensional $n\times n$ array $B$ -- to open the box -- in which $B[i,j]$ (for $i<j$) contains the sum of array elements $A[i]$ through $A[j]$, i.e., $B[i,j] = A[i]+A[i+1]+\dots+A[j]$. (The value of array element $B[i,j]$ is left unspecified whenever $i\geq j$, so it doesn't matter what the output is for these values.)

	Harry suggests the following simple algorithm to solve this problem:
	%
	\begin{small}
    	\begin{verbatim}
        	dwarvenLockBox(A) {
        	   for i = 1 to n {
        	      for j = i+1 to n {
        	         s = sum(A[i..j])       // look very closely here
        	         B[i,j] = s
        	}}}
    	\end{verbatim}
	\end{small}

	\begin{enumerate}
    	\item For some function $g$ that you should choose, give a bound of the form $\Omega(g(n))$ on the running time of this algorithm on an input of size $n$ (i.e., a bound on the number of operations performed by the algorithm).

    	\item For this same function $g$, show that the running time of the algorithm on an input of size $n$ is also $O(g(n))$. (This shows an asymptotically tight bound of $\Theta(g(n))$ on the running time.)
	\end{enumerate}

    \newpage

    % HARD PROBLEM
    \item (20 points) Consider the following strategy for choosing a pivot element for the {\tt Partition} subroutine of QuickSort, applied to an array $A$.
    \begin{itemize}
    	\item Let $n$ be the number of elements of the array $A$.
    	\item If $n\leq 24$, perform an Insertion Sort of $A$ and return.
    	\item Otherwise:
    	\begin{itemize}
    		\item Choose $2\lfloor n^{(1/2)} \rfloor$ elements at random from $n$; let $S$ be the new list with the chosen elements.
    		\item Sort the list $S$ using Insertion Sort and use the median $m$ of $S$ as a pivot element.
    		\item Partition using $m$ as a pivot.
    		\item Carry out QuickSort recursively on the two parts.
    	\end{itemize}
    \end{itemize}
    \begin{enumerate}
    	\item How much time does it take to sort $S$ and find its median? Give a $\Theta$ bound.
    	\item If the element $m$ obtained as the median of $S$ is used as the pivot, what can we say about the sizes of the two partitions of the array $A$?
    	\item Write a recurrence relation for the worst case running time of QuickSort with this pivoting strategy.
    \end{enumerate}

\end{enumerate}

\end{document}
